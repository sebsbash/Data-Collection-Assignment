\documentclass{article}

\usepackage[margin=1in]{geometry}
\usepackage{graphicx}

\begin{document}

		\title{A REPORT ABOUT CHALLENGES PEOPLE FACE AT WORK AND SOME PROPOSED SOLUTIONS }
		\author{Author :  LUTALO SEBASTIAN }
	           \date{Reg no: 14/U/8764/EVE}
                      \date{Student No: 214013265}
                   
		\maketitle
	

	\tableofcontents


\section{Introduction}
In uganda today, many people yarn for work. this has rose the need for sensitizatioin such that our freinds, relatives or any other get to know the chalenges he or she might find in the field of work.

Through this report, data is collected from differnt fields of work for example BODA-BODA,  Small scale businesses, Chapati making, tax field, car washing, among others.

\section{PERFOMANCE: THE MEASUREMENR}

\subsubsection{Target population}
This report targets those people who wants to start up business or be employed. 
\subsubsection{Measurement period}
The measurement period for this baseline  report is May 21, 2017 to june 30, 2017.
\subsubsection{Data source}
The dat is collect from people who started businesses and those who are employed by people.
\subsubsection{measurement system analysis}
The strength of this repeort is that, the information was collected with evidence from the people who are currently in the field of work and the limitations were, some peopleare too mean to give their information.
\subsubsection{Target perfomance level}
2017 goal: To reduce ignowlence by 10%.

\section{Methodology}
The following are some of the methhords used to gather information.
\subsection{Physical Contact}
Through this i was able to meet evry individual in person and get reliable information. Audio recording was taken and information was given without hesitation.
\includegraphics[width=6cm, height=8cm]{sfive} 

\subsection{Questionaires}
While using ODK Collect as a tool, i was able to design a questionaire that was followed while getting information from my interviewee
\includegraphics[width=6cm, height=8cm]{sone} 
\includegraphics[width=6cm, height=8cm]{stwo} 
\includegraphics[width=6cm, height=8cm]{sthree} 

\section{Data Analysis and interpretation}

\subsubsection{result}.
Data was collected from the western gate of makerere university, and the bar graph below show the ease with which information was obtained from different people in their respectives towns.
\subsubsection{Bar Graph}

\includegraphics[width=15cm, height=6cm]{bar} 

\subsubsection{Pie-Chart}

\includegraphics[width=15cm, height=6cm]{pie} 


\subsubsection{Findings}.
Many people got into business or joined the working sector without knowledge about what they are going to face when they are  there. This report has come to bridge that gap, such that peole dont regret after knocking themselves. Who wouldn't want to be aware of what he or she is going to pass through.

The report has at a certain extent included the following analysis. 
\begin{enumerate}

\item Stratification Analysis –determines the extent of the problem for relevant factors.  The important stratification factors will vary with each problem, but most problems will have several factors.  This analysis seeks to develop a pareto chart for the important factors. The differences identified can assist in identifying a root cause.

\item Regression Analysis - The goal of regression analysis is to determine the values of parameters for a function that cause the function to best fit a set of data observations that you provide. The purpose of regression analysis is to improve our ability to predict the next "real world" occurrence of our dependent variable. Regression analysis may be defined as the mathematical nature of the association between two variables. The association is determined in the form of a mathematical equation. Such an equation provides the ability to predict one variable on the basis of the knowledge of the other variable. The variable whose value is to be predicted is called the dependent variable. The variable about which knowledge is available or can be obtained is called the independent variable. In other words, the dependent variable is dependent upon the value of independent variables.

\item Correlation Analysis- Correlation analysis is the statistical tool that we can use to describe the degree to which one variable is linearly related to another. Frequently, correlation analysis is used in conjunction with regression  analysis to measure how well the least squares line fits the data . Correlation analysis can also be used by itself, however, to measure the degree of association between two variables. 
\end {enumerate}
\section{Conclusion and recomendation}
As I conclude, I would like to notify you again that, there is and there shall be many challenges at work place. so if you get stack somme where some how at your work place, you mmay vist https://www.careerwise.mnscu.edu for information and know more challenges we face at work.

\section{APPENDICES.}
In Uganda today, life is so delicate and treasured, as data was collected from an interviewee, a snap shot and an audio recording was required to evidence the true source of he information.This was tedious and many people rejected to give information because of this. This brings me to the idear that whoeve sees a gap in this report and want to cover it, you have to take note of that. 

\begin{table}[]
\centering
\caption{Acknowlegemnts  goes tothe following. }
\label{my-label}
\begin{tabular}{|l|l|l|l|l|}
\hline
NAME & LOCATION & GENDER & JOB  \\ \hline
BLESSED AISHA      &  NAJJERA                  & FEMALE          &    SUPERVISOR         \\ \hline
KIVUMBI DAVIN              & KIVVULU                    &  MALE  & 	CHAPAT MAKER                 \\ \hline
KIMBUGWE IVAN               & KAKAJJO                 &  MALE  &    BODA-BODA            \\ \hline
MBABAZI PATIENCE      &  KIVVULU                & FEMALE   & RETAIL SHOP    \\ \hline
\end{tabular}
\end{table}

\end{document}